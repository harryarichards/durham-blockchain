\documentclass[12pt, a4paper]{article}

%Template Values
\newcommand{\course}{Blockchain Summative}
\newcommand{\lecturer}{Anonymous Marking Code: Z0973527}

%The Preamble - Commands that affect the entire document.

%Margins
\usepackage{amssymb, amsmath, amsthm}
\newcommand{\R}{\mathbb{R}}

\usepackage[top=1in, bottom=1in, left=0.6in, right=0.6in]{geometry}

\usepackage[english]{babel}
\usepackage[utf8]{inputenc}
\usepackage{fancyhdr}
\usepackage{bbm}
\usepackage{bbold}
\usepackage{multicol}
\usepackage{graphicx}
\usepackage{wrapfig}
\usepackage[export]{adjustbox}
\usepackage{caption}
\usepackage{textcmds}
\usepackage{pgfplots}
\usepackage{ltablex}
\usepackage{subcaption}
\usepackage{enumitem}

\usepackage{color}
\usepackage[urlcolor = blue]{hyperref}
\hypersetup{citecolor=blue}
\hypersetup{
	colorlinks=true,
	linktoc=all,
	linkcolor=black
}

%For chapter pages
\fancypagestyle{plain}{
	\fancyhead{}
	\fancyfoot[CO,CE]{Page \thepage}
	\renewcommand{\headrulewidth}{0.5pt} % Header rule's width
	\renewcommand{\footrulewidth}{0.5pt} % Header rule's width
}

\renewcommand\sectionmark[1]{\markboth{#1}{}}

\fancypagestyle{MainStyle}{
	%No Evens and Odds in report - only book
	\fancyfoot[LE]{}
	\fancyfoot[LO]{}
	\fancyfoot[RE]{}
	\fancyfoot[RO]{}
	\fancyfoot[CE]{Page \thepage}
	\fancyfoot[CO]{Page \thepage}
	\fancyhead[LE]{\course}
	\fancyhead[LO]{\course}
	\fancyhead[RE]{\lecturer}
	\fancyhead[RO]{\lecturer}
	\fancyhead[CE]{}
	\fancyhead[CO]{}
	\renewcommand{\headrulewidth}{0.4pt} % Header rule's width
	\renewcommand{\footrulewidth}{0.4pt} % Header rule's width

}

\begin{document}
\renewcommand\refname{Bibliography}
\pagestyle{MainStyle}
%Top matter - Title, Author, Date (Today by default)

%Sectioning
%Sections: Part (Numerals), Chapter, Section, Subsection, Subsubsection, Paragraph, Subparagraph. Also Appendicies (Letters)

%To change normal and contents title use \section[Contents heading]{Normal heading}

%By default Part Chapter and Section get numbers as x=3 in \setcounter{secnumdepth}{x}. This can be changed between 1-7. The numbering depth that occurs in the contents can be changed using \setcounter{tocdepth}{x}.
%Unnumberered sections use \section*{title} syntax

%SPECIAL SECTIONS

%\renewcommand{\contentsname}{NEW NAME}
%\listoffigures
%\renewcommand{\listfigurename}{NEW NAME}
%\listoftables
%\renewcommand{\listtablename}{NEW NAME}

%Levels of sections can be changed as such \renewcommand*{\toclevel@chapter}{-1} % Put chapter depth at the same level as \part.

\subsection*{Task 1}
\begin{enumerate}
\item My User ID is: \textbf{bzqs27}
\item The \textbf{block hash target} I calculated was the following:\\
\phantom{abc} \textbf{3e7fc180000000000000000000000000000000000000000000000000000}
\item In order to attain a valid block I used a \textbf{nonce value} of $2943088$.
\item The number of double hashes performed to achieve this nonce value is $2943089$.\\ 
The real-time taken to obtain the valid block is $63.70$ seconds.
\item Estimate of how long it would take your code to mine a block at:
\begin{enumerate}
\item The initial bitcoin difficulty, $1$: $92959.95$ seconds.\\
\phantom{a}The following is an equation for the expected time to mine a block:\\
\phantom{abc}$t = \frac{n}{R}$ where $n$ is the number of hashes and $R$ is the hash rate.\\
\phantom{a}The initial target defined at difficulty $1$ is\\
\phantom{abc}00000000FFFF0000000000000000000000000000000000000000000000000000\\ \phantom{a}This target has $4(8) = 32$ leading zeros, and as a result we would expect approximately \phantom{a}$2^{32}$ hashes to be performed.\\
\phantom{a}Since earlier, I performed $2943089$ in $63.70$ seconds, we have a hash rate of approximately \\ \phantom{a}$R = \frac{2943089}{63.70}$.\\
\phantom{a}Using the above formula, we have:\\
\phantom{abc} $t = 2^{32} \cdot  \frac{63.70}{2943089} = 92959.95$

\item The $2018$ peak bitcoin difficulty, $7,454,968,648,263$: $4.09 \cdot 10^{17}$ seconds.\\
\phantom{a}The target at difficulty $7,454,968,648,263$ is\\
\phantom{abc}00000000000000000025c1910000015800000000000000000000000000000000\\ \phantom{a}This target has $4(18) + 2 = 74$ leading zeros, and as a result we would expect approximately \phantom{a}$2^{74}$ hashes to be performed.\\
\phantom{a}Using the above calculated hash rate: \\
\phantom{a}$R = \frac{2943089}{63.70}$.\\
\phantom{a}As well as the above formula, we have:\\
\phantom{abc} $t = 2^{74} \cdot  \frac{63.70}{2943089} = 4.09 \cdot 10^{17}$
\end{enumerate}
\item The ECDSA public key used is the following:\\ 
\phantom{abc}\textbf{0d059e547e831b396f67b41e96bc9f2366dbf3d26b6bcd9ac268a3cff9e81625}\\
\phantom{abc}\textbf{aa38fc9cf387a752dd6d843a0afce44d9cd7ba56212f6f73ed019fcdc6889cf1}
\item The \textbf{signature} of the message `\textbf{Hello world}' is the following:\\ 
\phantom{abc}\textbf{b0d338cc3c5a02c9eecd202bf2407a259e2b99fa72e6f12d8f7baa83f676a39a}\\
\phantom{abc}\textbf{705a75ff0810d80387862b89f822b024fbacf69d7606d52dc0e602c5a0c86285}
\item The signature used in calculating the below hit value is:\\
\phantom{abc}\textbf{d9e232243bd09634968c9363af96576a5d09c66caa64095a738a50db2f2f3b2a}\\
\phantom{abc}\textbf{31258f5292e770fe16bdb73eba4921c010c5eedbef0398a434887688e2b79577}
\item The \textbf{hit value}: \textbf{d8c1a43cd397bd9b}\\
\phantom{a} We obtain the hit value by performing the following steps:
\begin{itemize}
\renewcommand\labelitemi{\tiny$\bullet$}
\itemsep0em
\item Sign the generation signature:\\ \phantom{abc} \textbf{9737957703d4eb54efdff91e15343266123c5f15aaf033292c9903015af817f1}\\
using our ECDSA private (signing) key,\\ \phantom{abc} \textbf{041bd9e2bd0dd67995f3ada692f38f5a8926d33b5fed7e0018de696522c9d09e}
\item The resulting signature is stated in (8). We hash the resulting signature (using SHA-256), to obtain the following:\\ \phantom{abc} \textbf{d8c1a43cd397bd9b8fdb96d80a874426c9a054997e212c90e4458227df79fd0f}
\item The hit value is then the first $8$ bytes of the above, since each hexadecimal character consists of $4$ bits, this is the first $16$ characters:\\ \phantom{abc} \textbf{d8c1a43cd397bd9b}
\end{itemize}
\item The time when I will be able to find a new block is $165541515.56$ seconds.\\
\phantom{a} We obtain the time in the following manner: 
\begin{itemize}
\renewcommand\labelitemi{\tiny$\bullet$}
\itemsep0em
\item Once our target value $T$ is greater than the hit value, we may find a new block.
\item The target value $T$ is  $T = T_b \cdot S \cdot B_e $, where $T_b$ is the base target, $B_e$ is the effective balance and $S$ is the time in seconds since the last block was found.\\
In our case, $T = 1229782938247303 \cdot S \cdot 67 $.
\item By the above, $T > \textit{hit value}$ occurs when  $1229782938247303 \cdot S \cdot 67 > \textit{hit value}$
\item The decimal value of our hit value is $15618945563852520859$.
\item So $T>S$ occurs when $S > \frac{\textit{15618945563852520859}}{1229782938247303 \cdot 67} = 189.56$. So we're able to find a new block immediately after $189.56$ seconds.
\item Since the timestamp of the previous block is $165541326$, the time we will be able to produce a new block is the sum $165541326 + 189.56 =  165541515.56$.
\end{itemize}
\end{enumerate}



\subsection*{Task 2}
\begin{enumerate}
\item My User ID is: \textbf{bzqs27}
\item Transaction links:
\begin{enumerate}
\item \url{https://www.blockchain.com/btc/tx/b17fc28c1dc15ca4b05bf4784419c145b14e79ece54c793656997984c4e46715}
\item \url{https://www.blockchain.com/btc/tx/4716f13815add4e071b38f5e0f9cb6ba0eab094c68a047bddb43a3d9b68da075}
\item \url{https://www.blockchain.com/btc/tx/f602ce0777e04345c4e30aae1a8af0942e422f89b2832a3cf7b3ba569b3eabcb}
\end{enumerate}
\item \begin{enumerate}
\item Link (a) presents a transaction at [Block \#67].\\
This transaction involves $50$ BTC (newly generated coins) being awarded to the address \textbf{17uWtnxcJdpH7pEXXmr1SLrohVxe18B3fD}. Since no other transactions occur in the block there are no fees, this is a very early transaction occuring at 2009-01-10 22:21:12. Since this is a CoinBase transaction it needs no particular, however the miner can encode arbitrary data - in this case  `04ffff001d0173'. The output script is the following\\
\\
PUSHDATA(65)[04ea258025b3a5fa9352ab84951c1141affb5d7e0364ab18ed65d7f970b9af1c\\d7f9550cb69a016480d825e4e112ba9f16c717ec3a421c02155de4425a2066d208] CHECKSIG\\
\\
This is a Pay To Pubkey, or P2PK script that locks an output to a public key (a simpler version of the more commonly used P2PKH). The PUSHDATA(65)[PUBLIC KEY] section of the script pushes the $65-byte$ public key to the user. To unlock it, you need only provide a valid signature. When the script runs the CHECKSIG opcode it compares the signature with the public key, and pushes a $1$ on the stack if it is valid. This ensures that only a miner with the correct public key would have access to the block.


\item Link (b) presents a transaction at [Block \#500067].\\
Similarly to the above this transaction also presents newly generated coins being awarded to an address. In this case $16.72698693$ BTC are awarded, at 2018-12-26 13:06:54. Since mid-2016 the award for mining a bitcoin has been $12.5$ BTC, thus the additional $4.22698693$ BTC are for fees, for other transactions in the block. This indicates there are a large number of transactions, as the fee is relatively large compared to the reward value. \\
\\
Since mid-2012, as a result of BIP 0034, the CoinBase script must start with a push of the height of the block, this is in order to prevent CoinBase transactions from having the same transaction ID. In this case we have that the CoinBase script hex is:\\
\\
0363a10741d68e2247eee3b541d68e22478976a12f4254432e544f502ffabe6d6d3081b836aaba4e\\c4a30706e4102b8d2ff056bc061059377dd1682e527605fcff8000000000000000f8000e6c1654010\\000000000.\\ 
\\By the protocol the first byte `03' indicates that the next $3$ bytes present the height of the block. This enforces that each transaction is unique. In our case these are, $63a107$, converting this from little-endian gives us $07a163$ and in turn converting this to decimal gives us $500067$ - the height of the block. The remainding string of hexadecimal is arbitrary, and essentially meaningless in our case.\\
\\
We have the following $2$ output scripts:\\
\\
DUP HASH160 PUSHDATA(20)[ba507bae8f1643d2556000ca26b9301b9069dc6b]\\ EQUALVERIFY CHECKSIG\\
\\
This is a  Pay To Pubkey Hash or P2PKH script , a more commonly used locking script than the aforementioned P2PK script. It follows similar principles to P2PK however, rather than using the miner's public key, it uses the hash of the miner's public key. P2PKH allows us to fund an arbitrary transaction, no matter how complicated, using a 20-byte hash. Pay-to-Pubkey-hash addresses are similarly a 20-byte hash of the public key. EQUALVERIFY returns True if its inputs the Total In/ Total Out of the transaction) are exactly equal, and if they are not equal it marks the transaction as invalid.\\
\\
RETURN PUSHDATA(36)[aa21a9edbaf8d06742c94e5d7dab23cbbe93e4757f02b19724913cf\\5abf820654c3bba76]\\
\\
This script pushes $36$-bytes, i.e. $72$ hexadecimal characters to the stack. In this case the data pushed is the Merkle root of the witness tree. This is part of Segregated Witness a protocol upgrade introduced in $2017$, which changes the way that data is stored. It enables a larger number of transactions within the $1MB$ blocks, $1MB$ was the maximum block size at the time.
\item Link (c) presents a transaction at [Block \#56476].\\
This transaction consists of a payment from address \textbf{3PbJsixkjmjzsjCpi4xAYxxaL5Nn\\xrbF9B} to address \textbf{3MBiG7WKXeYE8GRaYzXWeR3sr8aY171o3K} of $0.0299$ BTC. This transaction has multiple input and outputs (2 in each case). \\
\\
The address \textbf{3PbJsixkjmjzsjCpi4xAYxxaL5NnxrbF9B}, provides two inptus. One of which is a value of $0.02419667$ BTC, and another is $0.09040641$ BTC. Thus the total input value is $0.11460308$ BTC.\\
A value of $0.08120308$ BTC is sent back to the address \textbf{3PbJsixkjmjzsjCpi4xAYxxaL5N\\nxrbF9B}, I think of this as the original address receiving `change' for paying too much, and a value of $0.0299$ BTC is transferred to the address \textbf{3MBiG7WKXeYE8GRaYzXW\\eR3sr8aY171o3K}. The output value amounts to $0.11110308$ BTC, the remaining $0.0035$ BTC is spent as miner fees. Thus the total input value is equal to the total output value, which we require.\\
\\
The 2 inputs are both of the following form:\\
\\
\phantom{ab} ScriptSig: PUSHDATA(22)[0014f7bae6ee31d59e79da8e857fb63429637f9e0d57]\\
\phantom{ab} WItness: $02473044022056f21b60c6d7443546ceff68db83f16fd7bbf94\dots$\\
\\
This another segwit transaction, and as expected we see the traditional scriptSig alongside `Witness' encoding which is introduced by this protocol upgrade. Both input scripts are of the same form, simple with different `Witness' encodings.\\
\\
The 2 outputs are:
\\
\phantom{ab} HASH160 PUSHDATA(20)[d5d7b573229379eb66f0cb98c112b715b32e2731] EQUAL\\
\phantom{ab} HASH160 PUSHDATA(20)[f03e6bf9b389bbd5d5669ff55c4dba30de995535] EQUAL\\
\\
This output makes use of Pay-to-Script-Hash, here the $20$ bytes of data being pushed is a redeem script. P2SH allow transactions to be sent to a script hash (address starting with 3) instead of a public key hash (addresses starting with 1). When redeeming the values, a HASH160 is performed on the script used in the transaction, and this is compared with the script provided by the recipient. To spend bitcoins sent via P2SH, the recipient must provide a script matching the script hash and data which makes the script evaluate to true. This adds extra security to the transaction.
\end{enumerate}
\item The bitcoin-testnet address I made the transactions from is:\\
\phantom{ab}\textbf{mqCYbVBd7jYKtGLoJMbfwWyi1PFFrxuVGA}
\item \begin{itemize}
\renewcommand\labelitemi{\tiny$\bullet$}
\itemsep0em
\item The transaction ID in hex is:\\ \phantom{ab}\textbf{f9e69cf2326a23dd65942bfcc0b15db73d38d7abb1f0ea9778f018da054eabb6}

\item This is a hyerlink to the transaction:\\ \phantom{ab} \url{https://chain.so/tx/BTCTEST/f9e69cf2326a23dd65942bfcc0b15db73d38d7abb1f0ea9778f018da054eabb6}
\end{itemize}
\item \begin{itemize}
\renewcommand\labelitemi{\tiny$\bullet$}
\itemsep0em
\item The transaction ID in hex is:\\ \phantom{ab} \textbf{a83c94074a3c9205ca68361466ae6cf5197ad94b988c7c4be6f97b0618322312}
\item This is a hyerlink to the transaction:\\ \phantom{ab} \url{https://chain.so/tx/BTCTEST/a83c94074a3c9205ca68361466ae6cf5197ad94b988c7c4be6f97b0618322312}
\end{itemize}
\item The hex script used is: \textbf{6a4c06627a71733237}\\
\phantom{ab}The general format of this hex script is:\\
\phantom{abcd}$\textit{OP\_RETURN} + \textit{OP\_PUSHDATA1} + \textit{number of bytes} + \textit{encoded user ID}$.\\
\phantom{ab}Using \url{https://en.bitcoin.it/wiki/Script}, we see that `A standard way of\\ \phantom{ab}attaching extra data to transactions is to add a zero-value output with a scriptPubKey\\ \phantom{ab}consisting of $OP\_RETURN$ followed by exactly one pushdata operation'.\\
\phantom{ab}Thus I chose to use $\textit{OP\_RETURN}$ alongside $\textit{OP\_PUSHDATA1}$, the description of $\textit{OP\_PUSHDATA1}$\\ \phantom{ab}tells us the next byte contains the number of bytes to be pushed onto the stack. Since there\\ \phantom{ab}are $6$ characters in \textbf{bzqs27}, the next byte is the hex encoding of $6$, which is $06$.\\
\phantom{ab}The hex encoding my user ID is $627a71733237$, thus we have the following:\\
\phantom{ab}The hex code for $\textit{OP\_RETURN}$ is $6a$.\\
\phantom{ab}The hex code for $\textit{PUSHDATA1}$ is $4c$.\\
\phantom{ab}The number of bytes is $6$, which has hex code $06$.\\
\phantom{ab}The hex encoding of my user ID \textbf{bzqs27} is $627a71733237$.\\
\phantom{ab}Concatenating the above information together gives the above script, \textbf{6a4c06627a71733237}.

\end{enumerate}



\subsection*{Task 3}
\hspace{\parindent} Gold has an intrinsic value due to its properties as an element and its uses in industry. There is a finite amount of gold, and no one is making anymore. In times of economic uncertainty people often invest in gold due to its reliable reputation, increasing its value. Gold is a physical object and as a result, no one can steal it from miles away as they can with cryptocurrencies. Throughout history, gold and money have been seen as synonymous, and thus Gold is seen as a reliable and low-risk store of value. However, gold has a history of low capital gains, and its value rarely increases significantly.
\par Bitcoin utilises cryptographic techniques to ensure that bitcoins can't be copied or counterfeited - we can verify that a bitcoin is real in seconds. The number of bitcoins is capped at $21$ million, and many people expect that the value of bitcoin will increase as the number of bitcoins entering the system decreases. What happens to bitcoin when there is no more bitcoin left to mine is uncertain, and people are very wary about what will happen to their `virtual savings'. Bitcoin is seen as less universal than some currencies, as many countries refuse to use it and have even made it illegal. Bitcoin is reliant on 3rd party software for wallets and exchanges, and people/organisations are constantly trying to exploit vulnerabilities in these 3rd parties. Not only this but bitcoin exchanges are not regulated by the government and generally do not provide enough insurance and security to be used to store money in the same way as a bank.
\par NXT utilises `Proof of Stake’ meaning the hardware requirements to `forge' NXT are much less than to `mine’ Bitcoin. The only way of attaining NXT in a Proof of Stake system is from a number of people who already have them, initially $1$ billion NXT were created and distributed across $73$ investors. Many of these initial investors still hold significant proportions of the NXT and have huge influence over the system. Unlike Bitcoin, NXT does not rely on third parties and instead provides its own wallet software that uses cold storage. NXT facilitates a range of applications that make use of the blockchain, and it could secure its future as more than just a cryptocurrency. The future of NXT is also very uncertain due its low market cap.
\par All cryptocurrencies are seen as a financial risk, and in NXT and Bitcoin, this is supported by a history of massive fluctuations in value. Whilst an investment in Gold is unlikely to reap the huge financial gain that an investment in Bitcoin or NXT could, it is also less likely to produce the sudden losses that investments in cryptocurrency often do. Whilst many cryptocurrencies have become obsolete with a lack of backing and advancements in technology, Gold will always have some intrinsic value. As a result, I feel Gold is the best long-term investment, but the best investment on a case-to-case basis is dictated by the investment parameters.

\end{document}